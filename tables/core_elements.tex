
\begin{longtable}[]{@{}
  >{\raggedright\arraybackslash}p{(\columnwidth - 2\tabcolsep) * \real{0.4000}}
  >{\raggedright\arraybackslash}p{(\columnwidth - 2\tabcolsep) * \real{0.6000}}@{}}
\toprule\noalign{}
\begin{minipage}[b]{\linewidth}\raggedright
Element
\end{minipage} & \begin{minipage}[b]{\linewidth}\raggedright
Description
\end{minipage} \\
\midrule\noalign{}
\endfirsthead
\toprule\noalign{}


\begin{minipage}[b]{\linewidth}\raggedright
Element
\end{minipage} & \begin{minipage}[b]{\linewidth}\raggedright
Description
\end{minipage} \\


\midrule\noalign{}
\endhead
\endlastfoot
\texttt{statement} & A character string describing causal relations
using dagitty syntax. \\

\texttt{nodes} & A list containing the nodes in the model. \\

\texttt{parents\_df} & A table listing nodes, whether they are root
nodes or not, and the number and names of parents they have. \\

\texttt{parameters} & A vector of `true' parameters. \\

\texttt{parameter\_names} & A vector of names of parameters. \\

\texttt{parameter\_mapping} & A matrix mapping from parameters into data
types. \\

\texttt{parameter\_matrix} & A matrix mapping from parameters into
causal types. \\

\texttt{parameters\_df} & A data frame containing parameter
information. \\

\texttt{causal\_types} & A data frame listing causal types and the nodal
types that produce them. \\

\texttt{nodal\_types} & A list with the nodal types of the model. \\

\texttt{data\_types} & A list with all data types consistent with the
model. \\

\texttt{ambiguities\_matrix} & A matrix mapping from causal types into
data types. \\

\texttt{prior\_hyperparameters} & A vector of alpha values used to
parameterize Dirichlet prior distributions; optionally provide node
names to reduce output. \\

\texttt{prior\_distribution} & A data frame of the parameter prior
distribution. \\

\texttt{posterior\_distribution} & A data frame of the parameter
posterior distribution. \\

\texttt{type\_prior} & A matrix of type probabilities using priors. \\

\texttt{type\_posterior} & A matrix of type probabilities using
posteriors. \\

\texttt{prior\_event\_probabilities} & A vector of data (event)
probabilities given a single realization of parameters. \\

\texttt{posterior\_event\_probabilities} & A sample of data (event)
probabilities from the posterior. \\

\texttt{data} & A data frame with data that was provided to update the
model. \\

\texttt{stan\_summary} & A \texttt{stanfit} summary with processed
parameter names. \\

\texttt{stanfit} & An unprocessed \texttt{stanfit} object as generated
by Stan. \\

\texttt{stan\_warnings} & A list of warnings produced by Stan during
updating. \\

\bottomrule
\caption{Elements of a model that can be inspected using
\texttt{inspect()}.}\label{tbl-core}\tabularnewline
\end{longtable}
